        %%******************************************%%
        %%                                          %%
        %%        Modello di tesi di laurea         %%
        %%            di Andrea Giraldin            %%
        %%                                          %%
        %%             2 novembre 2012              %%
        %%                                          %%
        %%******************************************%%


% I seguenti commenti speciali impostano:
% 1. 
% 2. PDFLaTeX come motore di composizione;
% 3. tesi.tex come documento principale;
% 4. il controllo ortografico italiano per l'editor.

% !TEX encoding = UTF-8
% !TEX TS-program = pdflatex
% !TEX root = tesi.tex
% !TEX spellcheck = it-IT

% PDF/A filecontents
\RequirePackage{filecontents}
\begin{filecontents*}{\jobname.xmpdata}
  \Title{Document’s title}
  \Author{Author’s name}
  \Language{it-IT}
  \Subject{The abstract, or short description.}
  \Keywords{keyword1\sep keyword2\sep keyword3}
\end{filecontents*}

\documentclass[10pt,                    % corpo del font principale
               a4paper,                 % carta A4
               twoside,                 % impagina per fronte-retro
               openright,               % inizio capitoli a destra
               english,                 
               italian,                 
               ]{book}    

%**************************************************************
% Importazione package
%************************************************************** 

\PassOptionsToPackage{dvipsnames}{xcolor} % colori PDF/A

\usepackage{colorprofiles}

\usepackage[a-2b,mathxmp]{pdfx}[2018/12/22]
                                        % configurazione PDF/A
                                        % validare in https://www.pdf-online.com/osa/validate.aspx

%\usepackage{amsmath,amssymb,amsthm}    % matematica

\usepackage[T1]{fontenc}                % codifica dei font:
                                        % NOTA BENE! richiede una distribuzione *completa* di LaTeX

\usepackage[utf8]{inputenc}             % codifica di input; anche [latin1] va bene
                                        % NOTA BENE! va accordata con le preferenze dell'editor

\usepackage[english, italian]{babel}    % per scrivere in italiano e in inglese;
                                        % l'ultima lingua (l'italiano) risulta predefinita
                                        
\setcounter{secnumdepth}{4}
     
                                 
\usepackage{float}

\usepackage{bookmark}                   % segnalibri

\usepackage{caption}                    % didascalie

\usepackage{chngpage,calc}              % centra il frontespizio

\usepackage{csquotes}                   % gestisce automaticamente i caratteri (")

\usepackage{emptypage}                  % pagine vuote senza testatina e piede di pagina

\usepackage{epigraph}			% per epigrafi

\usepackage{eurosym}                    % simbolo dell'euro

%\usepackage{indentfirst}               % rientra il primo paragrafo di ogni sezione

\usepackage{graphicx}                   % immagini

\usepackage{hyperref}                   % collegamenti ipertestuali

\usepackage[binding=5mm]{layaureo}      % margini ottimizzati per l'A4; rilegatura di 5 mm

\usepackage{listings}                   % codici

\usepackage{microtype}                  % microtipografia

\usepackage{mparhack,fixltx2e,relsize}  % finezze tipografiche

\usepackage{nameref}                    % visualizza nome dei riferimenti                                      
\usepackage[font=small]{quoting}        % citazioni

\usepackage{subfig}                     % sottofigure, sottotabelle

\usepackage[italian]{varioref}          % riferimenti completi della pagina

\usepackage{booktabs}                   % tabelle                                       
\usepackage{tabularx}                   % tabelle di larghezza prefissata                                    
\usepackage{longtable}                  % tabelle su più pagine                                        
\usepackage{ltxtable}                   % tabelle su più pagine e adattabili in larghezza
\usepackage{colortbl}					% tabelle con righe colorate
\usepackage{enumitem}					% cambiare il margine degli elenchi nelle tabelle

\newcolumntype{C}[1]{>{\centering\let\newline\\\arraybackslash\hspace{0pt}}m{#1}}
\newcolumntype{R}[1]{>{\raggedleft\let\newline\\\arraybackslash\hspace{0pt}}m{#1}}
\newcolumntype{L}[1]{>{\raggedright\let\newline\\\arraybackslash\hspace{0pt}}m{#1}}

\definecolor{bugWhite}{HTML}{f8f3ed}
\definecolor{lighter-bugBlue}{HTML}{d5e8f6}

\usepackage[toc, acronym]{glossaries}   % glossario
                                        % per includerlo nel documento bisogna:
                                        % 1. compilare una prima volta tesi.tex;
                                        % 2. eseguire: makeindex -s tesi.ist -t tesi.glg -o tesi.gls tesi.glo
                                        % 3. eseguire: makeindex -s tesi.ist -t tesi.alg -o tesi.acr tesi.acn
                                        % 4. compilare due volte tesi.tex.

\usepackage[backend=biber,style=verbose-ibid,hyperref,backref]{biblatex}
                                        % eccellente pacchetto per la bibliografia; 
                                        % produce uno stile di citazione autore-anno; 
                                        % lo stile "numeric-comp" produce riferimenti numerici
                                        % per includerlo nel documento bisogna:
                                        % 1. compilare una prima volta tesi.tex;
                                        % 2. eseguire: biber tesi
                                        % 3. compilare ancora tesi.tex.
\usepackage{fancyvrb}
\usepackage{cjkutf8-ko}
\input{tesi-config}                     % file con le impostazioni personali

\begin{document}
%**************************************************************
% Materiale iniziale
%**************************************************************
\frontmatter
\input{inizio-fine/frontespizio}
\input{inizio-fine/colophon}
% !TEX encoding = UTF-8
% !TEX TS-program = pdflatex
% !TEX root = ../tesi.tex

%**************************************************************
% Sommario
%**************************************************************
\cleardoublepage
\phantomsection
\pdfbookmark{Sommario}{Sommario}
\begingroup
\let\clearpage\relax
\let\cleardoublepage\relax
\let\cleardoublepage\relax

\chapter*{Sommario}

Il presente documento descrive l’attività di stage svolta presso l’azienda Zero12 s.r.l. Lo stage è stato svolto alla conclusione del percorso di studi della laurea triennale in Informatica ed ha avuto la durata di circa trecento ore. L’obiettivo dello stage è stato l’integrazione dei servizi AWS di image recognition (Amazon Rekognition), automatic speech recognition e natural language understanding (Amazon Lex) all’interno di un’applicazione web serverless basata su AWS Lambda con lo scopo di facilitare l’inserimento dei dati. 

%\vfill
%
%\selectlanguage{english}
%\pdfbookmark{Abstract}{Abstract}
%\chapter*{Abstract}
%
%\selectlanguage{italian}

\endgroup			

\vfill


% !TEX encoding = UTF-8
% !TEX TS-program = pdflatex
% !TEX root = ../tesi.tex

%**************************************************************
% Ringraziamenti
%**************************************************************
\cleardoublepage
\phantomsection
\pdfbookmark{Ringraziamenti}{ringraziamenti}

\begin{flushright}{
	\slshape    
	``Let us light up the night, we shine in our own ways. \\
		Shine, dream, smile''} \\ 
	\medskip
    --- 방탄소년단
\end{flushright}


\bigskip

\begingroup
\let\clearpage\relax
\let\cleardoublepage\relax
\let\cleardoublepage\relax

\chapter*{Ringraziamenti}

\noindent \textit{Innanzitutto, vorrei esprimere la mia gratitudine al \profTitle \myProf, relatore della mia tesi, per l'aiuto e il sostegno fornitomi durante la stesura del lavoro.}\\

\noindent The best is yet to come...

%\noindent \textit{Desidero ringraziare tutta la mia famiglia per il sostegno e la forza che mi hanno trasmesso in questi anni, aiutandomi ad andare avanti e non mollare. In particolare grazie mamma per le parole di incoraggiamento, l'ottimismo }\\
%
%\noindent \textit{Ho desiderio di ringraziare poi i miei amici per tutti i bellissimi anni passati insieme e le mille avventure vissute.}\\
\bigskip

\noindent\textit{\myLocation, \myTime}
\hfill \myName

\endgroup


\input{inizio-fine/indici}
\cleardoublepage

%**************************************************************
% Materiale principale
%**************************************************************
\mainmatter
% !TEX encoding = UTF-8
% !TEX TS-program = pdflatex
% !TEX root = ../tesi.tex

%**************************************************************
\chapter{Introduzione}
\label{cap:introduzione}
%**************************************************************

\section{L'azienda}

\azienda è un'azienda informatica nata nel 2012 specializzata nello sviluppo di soluzioni cloud native, sempre in prima linea nel seguire l'evoluzione di questo paradigma tecnologico. \\
L'azienda è partner \gls{AWS} e si occupa di progettazione e sviluppo software Web e Mobile per clienti provenienti da ambiti molto diversificati. \\
L'obiettivo di \azienda è aiutare i propri clienti a definire percorsi di innovazione includendo le tecnologie più avanzate tra cui per esempio
il cloud ed il machine learning per l'analisi di linguaggio naturale, immagini, video e per fare previsioni. 

	\begin{figure}[H]
		\centering
		\includegraphics[width=5cm]{immagini/logo-zero12.png} \\
		\caption{\label{fig:logo_zero12} Logo di Zero12 s.r.l.}
	\end{figure}

%**************************************************************
\section{L'offerta di stage}
	Attualmente in azienda è presente una piattaforma denominata MariBa con lo scopo di registrare i risultati di gioco del personale a Mario Kart e calcetto balilla. L'inserimento di tali dati però è completamente manuale: ogni partita deve essere
	inizializzata con l'inserimento dei nickname di tutti i giocatori e, una volta conclusa, i risultati devono
	essere inseriti manualmente all'interno della piattaforma. L'idea dello stage è di semplificare l'inserimento di questi
	dati attraverso l'utilizzo di tecnologie \gls{AWS} per il riconoscimento automatico dei giocatori e per la registrazione dei risultati comunicandoli vocalmente alla piattaforma. \\
	Il progetto è stato proposto dall'azienda in occasione dell'evento Stage-it 2022 (logo in \autoref{fig:logo_stageit}) finalizzato all'incontro tra aziende e studenti.
	
	\begin{figure}[H]
		\centering
		\includegraphics[width=7cm]{immagini/stageit.png} \\
		\caption{\label{fig:logo_stageit} Logo dell'evento Stage-it 2022}
	\end{figure}

%**************************************************************
\section{Struttura del documento}

	\begin{description}
	    \item[{\hyperref[cap:descrizione-stage]{Il secondo capitolo}}] descrive il progetto di stage e la pianificazione delle attività;
	    
		\item[{\hyperref[cap:tecnologie]{Il terzo capitolo}}] definisce le tecnologie utilizzate durante lo stage;
		
		\item[{\hyperref[cap:applicazione]{Il quarto capitolo}}] descrive il funzionamento generale dell'applicativo integrato;
	    
	    \item[{\hyperref[cap:rekognition]{Il quinto capitolo}}] approfondisce lo sviluppo del sistema di image recognition;
	    
	    \item[{\hyperref[cap:lex]{Il sesto capitolo}}] approfondisce lo sviluppo del sistema di voice service;
	   
	    \item[{\hyperref[cap:conclusioni]{Nel settimo capitolo}}] sono descritte le conclusioni dell'esperienza di stage e gli obiettivi raggiunti.
	\end{description}




             % Introduzione
% !TEX encoding = UTF-8
% !TEX TS-program = pdflatex
% !TEX root = ../tesi.tex

%**************************************************************
\chapter{Descrizione dello stage}
\label{cap:descrizione-stage}
%**************************************************************

%**************************************************************
\section{Introduzione al progetto}
In \azienda è stata creata una piattaforma denominata MariBa con lo scopo di registrare i risultati di gioco del personale a Mario Kart e calcetto balilla. Tale piattaforma è dotata di un sistema di intelligenza artificiale che, in base ai giocatori (o alle coppie nel caso del calcetto), è in grado di predire il risultato del match di gioco. Il limite della
piattaforma attuale è che tutti i dati, dall'inizializzazione di una partita ai risultati finali, devono essere inseriti
manualmente. \\

Al fine di rendere più immediato l'inserimento dei dati si vuole evolvere la piattaforma includendo le seguenti funzionalità:
\begin{itemize}
	\item Sistema di \gls{image recognition} pe riconoscere i giocatori e ruoli durante la fase di inizializzazione della partita
	e formazione delle squadre;
	\item Servizio vocale per l'inserimento dei risultati dei match giocati.
\end{itemize}


%**************************************************************
\section{Obiettivi formativi}
	Gli obiettivi formativi dell'attività di stage sono i seguenti:
	\begin{itemize}
		\item Apprendere come sviluppare un applicativo web con controlli vocali;
		\item Apprendere come svolgere attività di integrazione con servizi di Machine Learning in ambito \gls{image recognition}, \gls{automatic speech recognition} (ASR) e \gls{natural language understanding} (NLU);
	\end{itemize}
%**************************************************************

\section{Requisiti}
Nel primo giorno di stage si è svolto un incontro con il tutor aziendale per definire in modo dettagliato i requisiti. Nel corso dello stage il livello di obbligatorietà di tali requisiti è variato in risposta alle esigenze dell'azienda. \\
 Di seguito viene riportata la versione finale dell'analisi effettuata.

	\subsection{Requisiti obbligatori}
		Di seguito vengono elencati i requisiti obbligatori:
		\begin{itemize}
			\item Sviluppo di un micro-servizio per le attività di \emph{face detection} e \emph{recognition};
			\item Sviluppo di un micro-servizio per le attività di controllo vocale via web
		\end{itemize}
	\subsection{Requisiti desiderabili}
		Di seguito vengono elencati i requisiti desiderabili:
		\begin{itemize}
			\item Integrazione di un nuovo gioco all'interno della piattaforma;
			\item Implementazione di una versione semplificata per l'inserimento dei dati di Mario Kart per l'utilizzo
			della piattaforma in occasione del Summit \gls{AWS} di Milano.
		\end{itemize}
	\subsection{Requisiti facoltativi}
		Di seguito vengono elencati i requisiti facoltativi:
		\begin{itemize}
			\item Sviluppo di una \gls{skill} Alexa con le stesse funzionalità del \gls{chatbot} vocale richiesto come requisito obbligatorio e integrato sulla piattaforma web.
		\end{itemize}

%**************************************************************
\section{Pianificazione}
La durata complessiva dello stage è stata di 8 settimane di lavoro a tempo pieno per un totale di circa 320 ore. \\

\noindent Secondo il piano di lavoro iniziale definito con l'azienda, le attività sono distribuite come segue:

\begin{center}
	
	\renewcommand{\arraystretch}{1.5}
	
		\centering
		\begin{longtable}{| C{2.5cm} | C{2cm} | L{7.2cm} |}
			
			\hline
			
			\rowcolor{lighter-bugBlue}
			\textbf{Durata in ore} & \textbf{Settimana} & \textbf{Descrizione} \\
			
			\hline
			
			40 & 1 &
			\begin{itemize}[leftmargin=*]
				\item Studio delle tecnologie necessarie.
			\end{itemize} \\
			
			\hline
			
			80 & 2, 3 &
			\begin{itemize}[leftmargin=*]
				\item Progettazione e sviluppo di un micro-servizio per attività di \emph{face detection} per 
				la creazione di squadre di gioco;
				\item Integrazione con la piattaforma esistente. 
			\end{itemize}  \\
			
			\hline
		
			
			80 & 4, 5 &
			\begin{itemize}[leftmargin=*]
				\item Progettazione e sviluppo di un micro-servizio per il controllo vocale;
				\item Integrazione con la piattaforma esistente. 
			\end{itemize}  \\
			 
			\hline
			
			80 & 6, 7 &
			\begin{itemize}[leftmargin=*]
				\item Sviluppo della skill Alexa per il controllo vocale e l'aggiornamento dei risultati;
				\item Integrazione con la piattaforma esistente. 
			\end{itemize}  \\
			
			\hline
			
			40 & 8 &
			\begin{itemize}[leftmargin=*]
				\item Testing e stesura della documentazione di progetto delle attività di sviluppo condotte nelle settimane precedenti.
			\end{itemize} \\
			
			\hline
			
			\rowcolor{lighter-bugBlue}
			\multicolumn{2}{ | c | }{\textbf{Totale ore: }} & 	\multicolumn{1}{  c | }{\textbf{320}}\\
			
			\hline
		
			
			\caption{Pianificazione delle attività}
		\end{longtable}
		
	
\end{center}
Durante il periodo di stage in azienda il piano di lavoro ha subito modifiche e di conseguenza la versione finale della pianificazione riportata nel \autoref{cap:conclusioni} diverge da quella qui presentata. Tali modifiche sono state effettuate in risposta alle esigenze e richieste dell'azienda. 
Durante tutta la durata del tirocinio sono stati effettuati stand-up giornalieri con il tutor aziendale in affiancamento per 
monitorare lo stato di avanzamento ed evidenziare eventuali problemi sorti.
             % Descrizione stage
% !TEX encoding = UTF-8
% !TEX TS-program = pdflatex
% !TEX root = ../tesi.tex

%**************************************************************
\chapter{Tecnologie e strumenti}
\label{cap:tecnologie}
%**************************************************************

In questo capitolo vengono presentate le tecnologie utilizzate durante lo stage.\\

%**************************************************************
\section{Tecnologie per il back-end}
	\subsection{Serverless Framework}
	\emph{Serverless Framework} è un \gls{framework} web che permette di costruire applicazioni \gls{serverless} basate sul concetto \gls{FaaS}. Esso permette di definire funzioni Lambda e infrastrutture \gls{AWS} utilizzando sintassi \gls{YAML}.
	
	\begin{figure}[H]
		\centering
		\includegraphics[width=5cm]{immagini/serverless.png} \\
		\caption{\label{fig:logo_serverless} Logo Serverless Framework}
	\end{figure}
	
	\subsection{Node.js}
	\emph{Node.js} è un ambiente runtime open source per l'esecuzione di codice \emph{JavaScript} all'esterno di browser web. Esso consente infatti di utilizzare \emph{JavaScript} come linguaggio di programmazione lato server. All'interno del progetto
	viene utilizzata la versione 12.x per compatibilità con il codice già presente.
	
	\begin{figure}[H]
		\centering
		\includegraphics[width=3.3cm]{immagini/nodejs.png} \\
		\caption{\label{fig:logo_node} Logo Node.js}
	\end{figure}
	
	\subsection{AWS Lambda}
	\emph{AWS Lambda} è un servizio di calcolo serverless che permette l'esecuzione di codice per qualsiasi tipo di applicazione o servizio back-end senza bisogno di gestire un'infrastruttura server. \emph{Lambda} gestisce le risorse di elaborazione scalando automaticamente in risposta alla potenza di calcolo richiesta. Il linguaggio utilizzato per lo
	sviluppo di funzioni \emph{Lambda} è \emph{Node.js v12.x}.
	
	\begin{figure}[H]
		\centering
		\includegraphics[width=2.3cm]{immagini/aws-lambda.png} \\
		\caption{\label{fig:logo_lambda} Logo AWS Lambda}
	\end{figure}
	
	\subsection{Amazon API Gateway}
	\emph{API Gateway} è un servizio Amazon che consente di creare \emph{API RESTful} per permettere una comunicazione bidirezionale in tempo reale tra applicazioni e servizi di back-end. Le \gls{API} definite nell'applicazione sviluppata sono state
	integrate alle rispettive funzioni \emph{Lambda}.
	
	\begin{figure}[H]
		\centering
		\includegraphics[width=2.2cm]{immagini/api-gateway.png} \\
		\caption{\label{fig:logo_apigateway} Logo Amazon API Gateway}
	\end{figure}
	
	\subsection{Amazon DynamoDB}
	\emph{DynamoDB} è un database \gls{NoSQL}, \gls{serverless}, completamente gestito che supporta l'inserimento di dati di tipo chiave-valore. Facendo parte della famiglia di servizi messi a disposizione da Amazon, \emph{DynamoDB} si integra senza
	difficoltà con tutti i servizi \gls{AWS} e Amazon.
	
	\begin{figure}[H]
		\centering
		\includegraphics[width=2.5cm]{immagini/DynamoDB.png} \\
		\caption{\label{fig:logo_dynamoDB} Logo Amazon DynamoDB}
	\end{figure}
	
	\subsection{Amazon S3}
	\emph{Amazon Simple Storage Service} (S3) è un servizio di archiviazione di oggetti, scalabile, sicuro e con ottime prestazioni. Al suo interno i dati sono organizzati in \emph{bucket}. All'interno di ogni \emph{bucket} è possibile definire dei prefissi per poter organizzare al meglio gli oggetti caricati. \\
	All'intero del progetto questo servizio è stato utilizzato per effettuare l’hosting della web app e per il trasferimento indiretto di immagini e audio tra front-end e back-end.
	
	\begin{figure}[H]
		\centering
		\includegraphics[width=2.4cm]{immagini/amazon-s3.png} \\
		\caption{\label{fig:logo_s3} Logo Amazon S3}
	\end{figure}

	
	\subsection{Amazon Rekognition}
	\emph{Amazon Rekognition} è un software \emph{cloud-based} che mette a disposizione capacità di visione artificiale pre-addestrate e personalizzabili per estrarre informazioni dettagliate da immagini e video. Alcuni esempi di utilizzo sono la moderazione di contenuti e \emph{sentiment analysis}.\\ 
	All'interno del progetto è stato utilizzato per implementare la ricerca di volti all'interno di fotografie e per il loro riconoscimento in fase di inizializzazione di una partita e inserimento dei giocatori.
	
	\begin{figure}[H]
		\centering
		\includegraphics[width=2.5cm]{immagini/rekognition.png} \\
		\caption{\label{fig:logo_rekognition} Logo Amazon Rekognition}
	\end{figure}
	
	\subsection{Amazon Lex}
	\emph{Amazon Lex} è un servizio di intelligenza artificiale completamente gestito che mette a disposizione modelli avanzati di linguaggio naturale. Questo permette di sviluppare interfacce di comunicazione all'interno di applicazioni software. Nel progetto è stato utilizzato per implementare un \gls{chatbot} vocale in modo che 
	l'utente potesse interagire con MariBa e registrare i risultati delle partite giocate.
	
	\begin{figure}[H]
		\centering
		\includegraphics[width=2.4cm]{immagini/lex.png} \\
		\caption{\label{fig:logo_lex} Logo Amazon Lex}
	\end{figure}

\section{Tecnologie per il front-end}
	\subsection{TypeScript}
	\emph{TypeScript} è un linguaggio di programmazione sviluppato e manutenuto da Microsoft. \\
	Esso è un estensione del linguaggio di programmazione \emph{JavaScript}: utilizza la stessa sintassi ma con l'aggiunta del supporto alla tipizzazione e alle interfacce. 
	
	\begin{figure}[H]
		\centering
		\includegraphics[width=2cm]{immagini/typescript.png} \\
		\caption{\label{fig:logo_typescript} Logo TypeScript}
	\end{figure}

	\subsection{Angular}
	\emph{Angular} è un \gls{framework} open-source sviluppato da Google. Esso permette lo sviluppo di applicazioni web organizzate in componenti attraverso l'utilizzo di \emph{TypeScript}, \gls{HTML} e \gls{CSS}.
	
	\begin{figure}[H]
		\centering
		\includegraphics[width=2.5cm]{immagini/angular.png} \\
		\caption{\label{fig:logo_angular} Logo Angular}
	\end{figure}

	\subsection{Nebular}
	\emph{Nebular} è una libreria di \emph{Angular} gratuita e open-source per la creazione di interfacce utente.
	
	\begin{figure}[H]
		\centering
		\includegraphics[width=1.8cm]{immagini/nebular.png} \\
		\caption{\label{fig:logo_nebular} Logo Nebular}
	\end{figure}


\section{Strumenti di supporto a progettazione e codifica}
	\subsection{Git}
	\emph{Git} è un sistema di \gls{controllo di versione} distribuito. \emph{Git} permette di tenere traccia di tutte le modifiche avvenute all'interno di un progetto o di un singolo file e associa a ciascuna di esse il relativo autore. Permette inoltre di tornare ad una versione precedente del software eliminando le modifiche effettuate successivamente allo stato desiderato. Tutto ciò rende più semplice la collaborazione tra sviluppatori nella stesura del codice durante la fase di sviluppo software.
	
	\begin{figure}[H]
		\centering
		\includegraphics[width=2cm]{immagini/git.png} \\
		\caption{\label{fig:logo_git} Logo Git}
	\end{figure}

	\subsection{AWS CodeCommit}
	\emph{CodeCommit} è un servizio gestito, altamente scalabile e sicuro che consente l'hosting di \emph{repository} \emph{Git} privati. Esso custodisce i \emph{repository} nel cloud \gls{AWS} e supporta tutti i comandi \emph{Git}. Si è scelto di utilizzare \emph{CodeCommit} rispetto ad altri servizi equivalenti per compatibilità con la scelta aziendale e con il progetto esistente.
	
	\begin{figure}[H]
		\centering
		\includegraphics[width=2cm]{immagini/codecommit.png} \\
		\caption{\label{fig:logo_codecommit} Logo AWS CodeCommit}
	\end{figure}

	\subsection{VisualStudio Code}
\emph{Visual Studio Code} (VS Code) è un editor per il codice sorgente sviluppato da Microsoft. Esso possiede un controllo per \emph{Git} integrato e mette a disposizione numerose estensioni per facilitare la stesura del codice. Un esempio è \emph{Prettier}, estensione che automatizza la formattazione del codice in modo da mantenerlo ordinato e con uno stile consistente.
	
	\begin{figure}[H]
		\centering
		\includegraphics[width=2cm]{immagini/visual-studio-code.png} \\
		\caption{\label{fig:logo_vscode} Logo Visual Studio Code}
	\end{figure}

	\subsection{Balsamiq Wireframes}
	\emph{Balsamiq Wireframes} è uno strumento grafico per la creazione di schizzi per interfacce utente e schermate (\emph{wireframes}) di siti web e applicazioni. Durante lo stage è stata utilizzata la versione cloud. I \emph{wireframes} creati sono stati revisionati dal tutor aziendale, il quale ha potuto inserire commenti sulle modifiche da apportare. 
	
		\begin{figure}[H]
		\centering
		\includegraphics[width=2cm]{immagini/balsamiq.png} \\
		\caption{\label{fig:logo_balsamiq} Logo Balsamiq}
	\end{figure}
	










             % tecnologie e strumenti
% !TEX encoding = UTF-8
% !TEX TS-program = pdflatex
% !TEX root = ../tesi.tex

%**************************************************************
\chapter{Descrizione dell'applicativo esistente}
\label{cap:applicazione}
%**************************************************************

In questo capitolo viene descritto l'applicativo già esistente che è stato esteso durante lo stage.\\

\section{Architettura serverless}
	\subsection{Definizione di una tabella DynamoDB}
	\subsection{Definizione di una funzione Lambda}
	\subsection{Deploy del back-end}

\section{Web-App}
	\subsection{Funzionalità disponibili}
	\subsection{Deploy del front-end}             % descrizione dell'applicativo esistente
% !TEX encoding = UTF-8
% !TEX TS-program = pdflatex
% !TEX root = ../tesi.tex

%**************************************************************
\chapter{Integrazione di Amazon Rekognition}
\label{cap:rekognition}
%**************************************************************

In questo capitolo viene approfondito lo sviluppo e l'integrazione del sistema di image recognition\\

%**************************************************************

\section{Presentazione del problema}
MariBa è una piattaforma per il salvataggio dei risultati delle partite giocate dai dipendenti di \azienda durante le 
pause pranzo e caffè. \\
L'inizializzazione di una partita prevede l'inserimento dei nomi di tutti i giocatori, specificando anche la posizione (attacco o difesa) e le squadre nel caso del calcetto balilla. Questo procedimento di inserimento dei dati iniziali
risultava laborioso ed è stata quindi espressa la necessità di velocizzare tale procedura. \\
L'idea è quella di scattare una foto ai giocatori partecipanti in modo tale che l'applicazione li riconosca in modo automatico.

\section{Progettazione}
Di seguito viene descritta la progettazione ed il funzionamento della funzionalità di riconoscimento facciale implementata.
	\subsection{Architettura}
	Di seguito vengono descritte in modo approfondito le componenti sviluppate per il servizio di image recognition, i servizi utilizzati e come essi lavorino insieme per il raggiungimento dell'obiettivo. \\
	
	\subsubsection{Amazon Rekognition}
	
	\emph{Amazon Rekognition} è un software \emph{cloud-based} che mette a disposizione capacità di visione artificiale pre-addestrate e personalizzabili per estrarre informazioni dettagliate da immagini e video. Nel caso del progetto è stato utilizzato per indicizzare le facce e permetterne quindi il riconoscimento. In particolare, tutte le facce sono state salvate all'interno di una raccolta (\emph{collection}) e a ciascuna di esse è stato assegnato un ID univoco (\emph{faceId}). Per effettuare un riconoscimento si procede ad una ricerca di eventuali corrispondenze all'interno di tale \emph{collection} . \\
	Di seguito sono elencate le funzioni utilizzate:
	\begin{itemize}
		\item \texttt{DetectFaces}: individua le cento facce di dimensione maggiore presenti nell'immagine. Per ogni viso individuato ne restituisce i dettagli, in particolare la \emph{bounding box};
		\item \texttt{IndexFaces}: Individua i visi all'interno di un'immagine e li aggiunge ad una \emph{collection} specificata. Per questioni di sicurezza \emph{Rekognition} non salva direttamente l'immagine contenente la faccia ma ne salva solamente le caratteristiche che ne permettano il riconoscimento.
		\item \texttt{SearchFacesByImage}: data un'immagine, vengono identificate le facce presenti e successivamente ne vengono cercate delle corrispondenze all'interno di una \emph{collection} specificata.
		
	\end{itemize} 
	
	\subsubsection{S3}
	descrizione del bucket con la lifecycle rule
	
	\emph{Amazon Simple Storage Service} (S3) è un servizio di archiviazione oggetti. Al suo interno i dati sono organizzati in \emph{bucket}. All'interno di ogni \emph{bucket} è possibile definire dei prefissi per poter organizzare al meglio gli oggetti caricati. \\ Per evitare un passaggio diretto delle immagini tra front-end e back-end si è utilizzato questo servizio. S3 infatti fornisce la possibilità di generare un URL per effettuare operazione in una specifica posizione all'interno del \emph{bucket}.
	
	\subsubsection{AWS Lamda}
	descrizione delle lambda e di come vengono utilizzate
	
	
	
	\subsection{Funzionamento generale}
	% grafico funzionamento
	\begin{itemize}
		\item front-end richiede url per caricamento
		\item caricamento su s3
		\item chiamata callRekognition
		\item scarica l'immagine e chiama elaborateImage
		\item chiama rekognition.detectFaces per l'individuazione dei visi
		\item per ciascuno dei visi chiama rekognition.searchFaceByImage per il riconoscimento 
		\item per i visi riconosciuti restituisce il faceid della faccia il cui match ha maggiore confidenza
		\item per i visi non riconosciuti restituisce il crop della foto per permettere un'eventuale indicizzazione in caso di richiesta
		\item ritorna i dati a callRekognition
		\item callRekognition torna al frontend i dati 
		\item il front-end li visualizza
		
		\item Se viene richiesto di registrare un nuovo utente:\\
		il crop viene indicizzato nella collection di rekognition\\
		viene restituito il faceId del viso indicizzato \\
		il nuovo utente viene registrato con il faceId ricevuto in modo che la volta successiva possa essere riconosciuto direttamente
		
		\item Se viene richiesto di associare un nickname già esistente ad un viso:
		il crop della foto viene indicizzato\\
		viene restituito il faceId\\
		viene aggiornato il record dell'utente selezionato aggiungendo o sostituendo il faceId
	\end{itemize}
	
		
		
	
	
	\subsection{Design dell'interfaccia}
	Per il design dell'interfaccia, prima della sua effettiva codifica, sono stati realizzati dei \emph{wireframes} utilizzando \textbf{Balsamiq} che definissero il flusso di funzionamento a livello di front-end. \\ 

	\noindent Successivamente, la realizzazione dell'interfaccia è avvenuta utilizzando \textbf{Angular 12.x} in combinazione con la libreria 
	\textbf{Nebular}, specifica per lo sviluppo di interfacce utente. \\
	Il procedimento di inserimento dei giocatori è leggermente differente a seconda del gioco utilizzato.
	
		\subsubsection{Inserimento di una nuova partita di calcetto}
		Nella pagina per l'inizializzazione di una nuova partita di calcetto è stato aggiunto un pulsante per attivare la webcam e scattare la foto  contenente i volti dei giocatori (\autoref{fig:calcetto-1}). Lo scatto viene quindi inviato a \emph{Amazon Rekognition} per il riconoscimento. 
		
		\begin{figure}[H]
			\centering
			\includegraphics[width=\textwidth]{immagini/calcetto-1.png} \\
			\caption{\label{fig:calcetto-1} Schermata iniziale di calcetto}
		\end{figure}
	
		\noindent Una volta ricevuto il risultato dell'elaborazione, i giocatori vengono visualizzati in card selezionabili per la formazione delle squadre (\autoref{fig:calcetto-2}). I giocatori possono appartenere a tre categorie differenti:
		\begin{itemize}
			\item \textbf{Giocatore riconosciuto}: viene mostrato il nickname corrispondente al volto riconosciuto;
			\item \textbf{Giocatore non riconosciuto ma registrato}: quando selezionato richiede l'associazione del nickname al viso scegliendo tra i nickname già presenti nel database; 
			\item\textbf{Giocatore non riconosciuto e non registrato}: quando selezionato richiede la registrazione di un nuovo giocatore; 
		\end{itemize}
	
		\noindent Nel caso in cui uno o più giocatori non fossero presenti all'interno della fotografia scattata, è possibile inserirli manualmente. Una volta inserito il nickname desiderato, viene mostrata la card corrispondente. \\
		Il numero di giocatori selezionabili per ciascuna squadra è esattamente due. 
		
		\begin{figure}[H]
			\centering
			\includegraphics[width=\textwidth]{immagini/calcetto-2.png} \\
			\caption{\label{fig:calcetto-2} Selezione dei giocatori}
		\end{figure}
		
		\noindent Dopo la selezione dei giocatori sarà possibile scambiare i due nickname all'interno di ciascuna squadra per associare loro il ruolo desiderato e formare le squadre definitive (\autoref{fig:calcetto-3}) .
			
		\begin{figure}[H]
			\centering
			\includegraphics[width=\textwidth]{immagini/calcetto-3.png} \\
			\caption{\label{fig:calcetto-3} Formazione squadre di calcetto}
		\end{figure}
		
		
		\subsubsection{Inserimento di una nuova partita di Mario Kart e Duck Game}
		Il procedimento da seguire per l'inserimento di una partita di Mario Kart o di Duck Game tramite l'utilizzo del riconoscimento facciale dei giocatori è pressoché il medesimo. \\
		La sola differenza tra i due giochi è individuabile nel numero di giocatori selezionabili: 
		\begin{itemize}
			\item per Mario Kart devono essere inseriti esattamente quattro giocatori (\autoref{fig:kart-1});
			\item per Duck Game si possono inserire da un minimo di due ad un massimo di otto giocatori (\autoref{fig:duck-1}).
		\end{itemize}
	
		Ad entrambe le schermate, come per calcetto, è stato quindi aggiunto un pulsante per attivare la webcam e scattare la fotografia da inviare ad \emph{Amazon Rekognition}.
		
		\begin{figure}[H]
			\centering
			\includegraphics[width=\textwidth]{immagini/duck-1.png} \\
			\caption{\label{fig:duck-1} Schermata iniziale di Duck Game}
		\end{figure}
	
		\begin{figure}[H]
			\centering
			\includegraphics[width=\textwidth]{immagini/kart-1.png} \\
			\caption{\label{fig:kart-1} Schermata iniziale di Mario Kart}
		\end{figure}
		
		\noindent Una volta ricevuto il risultato dell'elaborazione, i giocatori vengono visualizzati in card selezionabili per la formazione delle squadre (\autoref{fig:kart-2}). I giocatori possono appartenere a tre categorie differenti:
		\begin{itemize}
			\item \textbf{Giocatore riconosciuto}: viene mostrato il nickname corrispondente al volto riconosciuto;
			\item \textbf{Giocatore non riconosciuto ma registrato}: il giocatore è già registrato ma non ha ancora volto associato. Quando selezionato richiede l'associazione del nickname al viso scegliendo tra i nickname già presenti nel database; 
			\item\textbf{Giocatore non riconosciuto e non registrato}: quando selezionato richiede la registrazione di un nuovo giocatore; 
		\end{itemize} 
	
		\noindent Nel caso in cui uno o più giocatori non fossero presenti all'interno della fotografia scattata, è possibile inserirli manualmente. Una volta inserito il nickname desiderato, viene mostrata la card corrispondente.
		
		\begin{figure}[H]
			\centering
			\includegraphics[width=\textwidth]{immagini/kart-2.png} \\
			\caption{\label{fig:kart-2} Selezione dei giocatori di Mario Kart}
		\end{figure}
	
		
				% sviluppo image recognition
% !TEX encoding = UTF-8
% !TEX TS-program = pdflatex
% !TEX root = ../tesi.tex

%**************************************************************
\chapter{Integrazione di Amazon Lex}
\label{cap:lex}
%**************************************************************

In questo capitolo viene approfondito lo sviluppo e l'integrazione del sistema di interazione vocale \\

%**************************************************************

\section{Presentazione del problema}
\section{Progettazione}
	\subsection{Architettura}
	\subsection{Design dell'interfaccia}

				% sviluppo lex

% !TEX encoding = UTF-8
% !TEX TS-program = pdflatex
% !TEX root = ../tesi.tex

%**************************************************************
\chapter{Conclusioni}
\label{cap:conclusioni}
%**************************************************************

%**************************************************************
\section{Consuntivo finale}

\begin{center}
	
	\renewcommand{\arraystretch}{1.5}
	
	\centering
	\begin{longtable}{| C{2.5cm} | C{2cm} | L{7.2cm} |}
		
		\hline
		
		\rowcolor{lighter-bugBlue}
		\textbf{Durata in ore} & \textbf{Settimana} & \textbf{Descrizione} \\
		
		\hline
		
		40 & 1 &
		\begin{itemize}[leftmargin=*]
			\item Studio delle tecnologie necessarie.
		\end{itemize} \\
		
		\hline
		
		80 & 2, 3 &
		\begin{itemize}[leftmargin=*]
			\item Progettazione e sviluppo di un micro-servizio per attività di \emph{face detection} per 
			la creazione di squadre di gioco;
			\item Integrazione con la piattaforma esistente. 
		\end{itemize}  \\
		
		\hline
		
		40 & 4 &
		\begin{itemize}[leftmargin=*]
			\item Progettazione e sviluppo di una sezione del sito per l'inserimento dei risultati di un nuovo gioco;
			\item Integrazione con la piattaforma esistente. 
		\end{itemize}  \\
		
		\hline
		
		100 & 5, 6, 7 &
		\begin{itemize}[leftmargin=*]
			\item Progettazione e sviluppo di un micro-servizio per il controllo vocale;
			\item Integrazione con la piattaforma esistente. 
		\end{itemize}  \\
		
		\hline 
		
		20 & 7 &
		\begin{itemize}[leftmargin=*]
			\item Progettazione e implementazione di una versione semplificata della pagina per l'inserimento delle 
			partite di Mario Kart da utilizzare in occasione del Summit \gls{AWS} di Milano.
		\end{itemize} \\
		
		\hline
		
		40 & 8 &
		\begin{itemize}[leftmargin=*]
			\item Stesura della documentazione di progetto delle attività di sviluppo condotte nelle settimane precedenti.
		\end{itemize} \\
		
		\hline
		
		\rowcolor{lighter-bugBlue}
		\multicolumn{2}{| c | }{\textbf{Totale ore: }} & 	\multicolumn{1}{  c | }{\textbf{320}}\\
		
		\hline
		
		
		\caption{Consuntivo finale}
	\end{longtable}
	
	
\end{center}

%**************************************************************
\section{Raggiungimento degli obiettivi}
Durante il progetto di stage sono stati soddisfatti tutti i requisiti obbligatori. \\
Il tempo pianificato per la formazione iniziale e la realizzazione del sistema di riconoscimento dei giocatori è stato adeguato. \\
Ha invece richiesto più tempo del previsto l'implementazione del \gls{chatbot} vocale, avendo incontrato alcune difficoltà nell'interpretazione non sempre corretta da parte del bot dei nickname dei giocatori pronunciati dall'utente.
Oltre a quelli classificati obbligatori, sono stati soddisfatti anche tutti i requisiti desiderabili. Sono state quindi aggiunte due pagine al sito: 
\begin{itemize}
	\item Una per l'inserimento dei risultati di gioco a Duck Game;
	\item Una per l'inserimento più rapido delle partite di Mario Kart attraverso riconoscimento facciale. Questa
	funzionalità ha permesso l'utilizzo del prodotto allo stand di \azienda in occasione del Summit \gls{AWS} tenutosi a Milano durante gli ultimi giorni di stage.
\end{itemize}

\noindent Per quanto riguarda invece i requisiti facoltativi, non vi è stato il tempo necessario alla loro implementazione. \\

\noindent Il raggiungimento degli obiettivi, rappresentato dal grado di implementazione dei requisiti, è schematizzato in \autoref{tab:tracciamento} e \autoref{tab:completamento}.

\begin{center}
	
	\renewcommand{\arraystretch}{1.5}
	
	\centering
	\begin{longtable}{| C{2.5cm} | C{4cm} |}
		
		\hline
		
		\rowcolor{lighter-bugBlue}
		\textbf{Requisito} &  \textbf{Stato} \\
		
		\hline
		
		\texttt{RO-1} & Implementato \\
		
		\hline 
		
		\texttt{RO-1.1} & Implementato \\
		
		\hline 
		
		\texttt{RO-1.2} & Implementato \\
		
		\hline 
		
		\texttt{RO-1.3} & Implementato \\
		
		\hline
		
		\texttt{RO-1.4} & Implementato  \\
		
		\hline 
		
		\texttt{RO-1.5} & Implementato \\
		
		\hline 
		
		\texttt{RO-1.6} & Implementato  \\
		
		\hline 
		
		\texttt{RO-2} &Implementato  \\
		
		\hline 
		
		\texttt{RO-2.1} & Implementato  \\
		
		\hline 
		
		\texttt{RO-2.2} & Implementato  \\
		
		\hline
		
		\texttt{RD-1} & Implementato  \\
		
		\hline 
		
		\texttt{RD-1.1} &Implementato  \\
		
		\hline 
		
		\texttt{RD-1.2} & Implementato \\
		
		\hline 
		
		\texttt{RD-1.3} & Implementato  \\	
	
		\hline 
		
		\texttt{RD-2} & Implementato  \\
		
		\hline 
		
		\texttt{RF-1} & Non Implementato \\
		
		\hline 
		
		\caption{Tracciamento dei requisiti}\label{tab:tracciamento}
	\end{longtable}
	
	
\end{center}

\begin{center}
	
	\renewcommand{\arraystretch}{1.5}
	
	\centering
	\begin{longtable}{| C{2.5cm} | C{4cm} |}
		
		\hline
		
		\rowcolor{lighter-bugBlue}
		\textbf{Requisiti} &  \textbf{Completamento} \\
		
		\hline
		
		\textbf{Obbligatori} & 100\% \\
		
		\hline 
		
		\textbf{Desiderabili} & 100\% \\
		
		\hline 
		
		\textbf{Facoltativi} & 0\% \\
		
		\hline 
		
		\caption{Completamento dei requisiti}\label{tab:completamento}
	\end{longtable}
	
	
\end{center}

%**************************************************************
\section{Conoscenze acquisite e valutazione personale}
Durante le otto settimane di stage ho avuto modo di apprendere moltissimo, sia a livello di hard-skill che di soft-skill.

\noindent Ho avuto modo non solo di studiare tecnologie che non avevo mai utilizzato (come ad esempio \gls{AWS} e \emph{Angular}), ma anche di approfondire e affinare le conoscenze che durante questi anni di università avevo acquisito.\\

\noindent Ho imparato inoltre a comunicare in modo più efficace non solo le mie idee ma anche le mie difficoltà, in modo che altri potessero aiutarmi e consigliarmi. Mi ha aiutato anche a comprendere meglio il funzionamento corretto e proficuo di un rapporto collaborativo tra colleghi con lo scopo i raggiungere un obiettivo comune.

\noindent Questo progetto di stage mi ha permesso di sperimentare, mettermi alla prova e utilizzare la mia creatività per 
realizzare un prodotto finale che incontrasse le aspettative dell'azienda. \\

\noindent Valuto dunque molto positivamente l'esperienza di stage svolta presso \azienda poiché mi ha permesso di entrare in contatto con un ambiente lavorativo reale, di mettere a frutto le mie conoscenze pregresse e svilupparne di nuove.
             % Conclusioni


%**************************************************************
% Materiale finale
%**************************************************************
\backmatter
\printglossaries
\input{inizio-fine/bibliografia}
\end{document}
