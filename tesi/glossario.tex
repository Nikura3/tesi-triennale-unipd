%**************************************************************
% Glossario
%**************************************************************
%\renewcommand{\glossaryname}{Glossario}

\newglossaryentry{API}
{
	name=API,
	text=Application Program Interface,
	sort=api,
	description={in informatica con il termine \emph{Application Programming Interface API} (ing. interfaccia di programmazione di un'applicazione) si indica ogni insieme di procedure disponibili al programmatore, di solito raggruppate a formare un set di strumenti specifici per l'espletamento di un determinato compito all'interno di un certo programma. La finalità è ottenere un'astrazione, di solito tra l'hardware e il programmatore o tra software a basso e quello ad alto livello semplificando così il lavoro di programmazione}
}



\newglossaryentry{AWS}
{
	name=AWS,
	description={azienda di proprietà Amazon che fornisce servizi di cloud computig, disponibili sull'omonima piattaforma}
}

\newglossaryentry{serverless}
{
	name=serverless,
	description={modello di esecuzione cloud che non richiede la presenza di server fisici che devono essere mantenuti e configurati. I server sono comunuque presetni ma si trovano nel cloud e allocano risorse dinamicamente appena queste vengono richieste. Quando l'applicazione non è in uso, nessuna risorsa viene consumata ed il prezzo risulta quindi essere pari a zero}
}

\newglossaryentry{framework}
{
	name=framework,
	description={architettura logica riutilizzabile sulla quale un software può essere progettato. È definito da un insieme di classi astratte e dalle relazioni tra di esse e può includere programmi, librerie e strumenti di supporto}
}

\newglossaryentry{FaaS}
{
	name=FaaS,
	description={categoria dello sviluppo cloud che fornisce una piattaforma che permetta agli sviluppatori di eseguire e gestire le varie funzionalità della propria applicazione senza necessità di gestire e mantenere infrastrutture fisiche. \emph{FaaS} è una delle modalità con cui si può costruire un'architettura \gls{serverless}}
}

\newglossaryentry{YAML}
{
	name=YAML,
	description={con l’acronimo ricorsivo (\emph{YAML Ain’t a Markup Language}) formato per la serializzazione di dati leggibile da esseri umani. Questo formato può essere utilizzato come file di configurazione, come nel caso del \emph{Serverless Framework}}
}

\newglossaryentry{automatic speech recognition}
{
	name=Automatic Speech Recognition,
	description={prova}
}

\newglossaryentry{natural language understanding}
{
	name=Natural Language Understanding,
	description={prova}
}

\newglossaryentry{skill}
{
	name=skill,
	description={prova}
}

\newglossaryentry{chatbot}
{
	name=chatbot,
	description={prova}
}

\newglossaryentry{NoSQL}
{
	name=NoSQL,
	description={prova}
}

\newglossaryentry{prefisso}
{
	name=prefisso,
	description={prova}
}

\newglossaryentry{deploy}
{
	name=deploy,
	description={prova}
}

\newglossaryentry{CSS}
{
	name=CSS,
	description={prova}
}

\newglossaryentry{HTML}
{
	name=HTML,
	description={prova}
}

\newglossaryentry{controllo di versione}
{
	name=controllo di versione,
	description={prova}
}

\newglossaryentry{repository}
{
	name=repository,
	description={prova}
}

\newglossaryentry{CORS}
{
	name=CORS,
	description={prova}
}