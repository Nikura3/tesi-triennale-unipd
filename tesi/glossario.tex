%**************************************************************
% Glossario
%**************************************************************
%\renewcommand{\glossaryname}{Glossario}

\newglossaryentry{API}
{
	name=API,
	text=Application Program Interface,
	sort=api,
	description={in informatica con il termine \emph{Application Programming Interface API} (ing. interfaccia di programmazione di un'applicazione) si indica ogni insieme di procedure disponibili al programmatore, di solito raggruppate a formare un set di strumenti specifici per l'espletamento di un determinato compito all'interno di un certo programma. La finalità è ottenere un'astrazione, di solito tra l'hardware e il programmatore o tra software a basso e quello ad alto livello semplificando così il lavoro di programmazione}
}



\newglossaryentry{AWS}
{
	name=AWS,
	description={prova}
}

\newglossaryentry{serverless}
{
	name=Serverless,
	description={prova}
}

\newglossaryentry{framework}
{
	name=Framework,
	description={prova}
}

\newglossaryentry{FaaS}
{
	name=FaaS,
	description={prova}
}

\newglossaryentry{YAML}
{
	name=YAML,
	description={prova}
}

\newglossaryentry{image recognition}
{
	name=Image Recogition,
	description={prova}
}

\newglossaryentry{automatic speech recognition}
{
	name=Automatic Speech Recognition,
	description={prova}
}

\newglossaryentry{natural language understanding}
{
	name=Natural Language Understanding,
	description={prova}
}

\newglossaryentry{skill}
{
	name=skill,
	description={prova}
}

\newglossaryentry{chatbot}
{
	name=chatbot,
	description={prova}
}

\newglossaryentry{NoSQL}
{
	name=NoSQL,
	description={prova}
}

\newglossaryentry{prefisso}
{
	name=prefisso,
	description={prova}
}

\newglossaryentry{deploy}
{
	name=deploy,
	description={prova}
}

\newglossaryentry{CSS}
{
	name=CSS,
	description={prova}
}

\newglossaryentry{HTML}
{
	name=HTML,
	description={prova}
}

\newglossaryentry{controllo di versione}
{
	name=controllo di versione,
	description={prova}
}

\newglossaryentry{repository}
{
	name=repository,
	description={prova}
}