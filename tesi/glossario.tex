%**************************************************************
% Glossario
%**************************************************************
%\renewcommand{\glossaryname}{Glossario}

\newglossaryentry{API}
{
	name=API,
	text=Application Program Interface,
	sort=api,
	description={in informatica con il termine \emph{Application Programming Interface API} (ing. interfaccia di programmazione di un'applicazione) si indica ogni insieme di procedure disponibili al programmatore, di solito raggruppate a formare un set di strumenti specifici per l'espletamento di un determinato compito all'interno di un certo programma. La finalità è ottenere un'astrazione, di solito tra l'hardware e il programmatore o tra software a basso e quello ad alto livello semplificando così il lavoro di programmazione}
}



\newglossaryentry{AWS}
{
	name=AWS,
	description={azienda di proprietà Amazon che fornisce servizi di cloud computig, disponibili sull'omonima piattaforma}
}

\newglossaryentry{serverless}
{
	name=serverless,
	description={modello di esecuzione cloud che non richiede la presenza di server fisici che devono essere mantenuti e configurati. I server sono comunuque presetni ma si trovano nel cloud e allocano risorse dinamicamente appena queste vengono richieste. Quando l'applicazione non è in uso, nessuna risorsa viene consumata ed il prezzo risulta quindi essere pari a zero}
}

\newglossaryentry{framework}
{
	name=framework,
	description={architettura logica riutilizzabile sulla quale un software può essere progettato. È definito da un insieme di classi astratte e dalle relazioni tra di esse e può includere programmi, librerie e strumenti di supporto}
}

\newglossaryentry{FaaS}
{
	name=FaaS,
	description={categoria dello sviluppo cloud che fornisce una piattaforma che permetta agli sviluppatori di eseguire e gestire le varie funzionalità della propria applicazione senza necessità di gestire e mantenere infrastrutture fisiche. \emph{FaaS} è una delle modalità con cui si può costruire un'architettura \gls{serverless}}
}

\newglossaryentry{YAML}
{
	name=YAML,
	description={con l’acronimo ricorsivo (\emph{YAML Ain’t a Markup Language}) formato per la serializzazione di dati leggibile da esseri umani. Questo formato può essere utilizzato come file di configurazione, come nel caso del \emph{Serverless Framework}}
}

\newglossaryentry{automatic speech recognition}
{
	name=Automatic Speech Recognition,
	description={campo dell'informatica che sviluppa metodologie e tecnologie che permettano il riconoscimento e 
		la trduzione di linguaggio parlato in testo in modo automatico attraverso l'utilizzo del computer}
}

\newglossaryentry{natural language understanding}
{
	name=Natural Language Understanding,
	description={campo dell'intelligenza artificiale che si occupa della comprensione del testo da parte delle macchine}
}

\newglossaryentry{skill}
{
	name=skill,
	description={equivalente di un'applicazione, sviluppata per l'assistente intelligente Alexa}
}

\newglossaryentry{chatbot}
{
	name=chatbot,
	description={software che simula conversazioni simili a quelle con esseri umani attraverso chat scritte o vocali. Il suo compito è quello di aiutare gli utenti a svolgere specifiche attività rispondendo nel modo corretto alle  richieste}
}

\newglossaryentry{NoSQL}
{
	name=NoSQL,
	description={tipologia di database che fornisce un meccanismo per salvare dati diverso dal concetto di
	\emph{tabella}, tipico dei database relazionali. Essi salvano i dati utilizzando strutture differenti, come ad
	esempio oggetti di tipo \emph{chiave-valore} o \emph{documento}}.
}

\newglossaryentry{prefisso}
{
	name=prefisso,
	description={In \emph{S3} un prefisso è una stringa di caratteri all'inizio del nome della chiave dell'oggetto. 
	Essi servono per organizzare i dati archiviati all'interno di un bucket, risultando infatti divisi in base al 
	prefisso posseduto. In generale, visivamente risulta simile al concetto di \emph{cartella}}
}

\newglossaryentry{deploy}
{
	name=deploy,
	description={procedura di rilascio di un sistema software o di un’applicazione}
}

\newglossaryentry{CSS}
{
	name=CSS,
	description={(\emph{Cascading Style Sheets)} è un linguaggio di stile utilizzato per definire la formattazione di 
	documenti \gls{HTML}, come ad esempio pagine web}
}

\newglossaryentry{HTML}
{
	name=HTML,
	description={\emph{(HyperText Markup Language)} è un linguaggio di markup che definisce le modalità di impaginazione
	del contenuto di pagine web attraverso l'utilizzo di \emph{tag}}
}

\newglossaryentry{controllo di versione}
{
	name=controllo di versione,
	description={sistema che registra nel tempo i cambiamenti di uno o più file, così da poter richiamare una specifica versione in un seconodo momento}
}

\newglossaryentry{CORS}
{
	name=CORS,
	description={\emph{(Cross-origin resource sharing)} meccanismo che permette di defnire restrizioni sulle risorse
	che possono essere richieste da domani diversi da quello su cui tali risorse sono allocate}
}

\newglossaryentry{CloudFront}
{
	name=CloudFront,
	description={Amazon CloudFront è un servizio di rete di distribuzione di contenuti (CDN). Esso fornisce una rete 
	globalmente distribuita di server che mantengono i contenuti (come video o media) in modo da aumentare la velocità di accesso a tali contentuti quando vengono richiesti}
}

\newglossaryentry{distanza di Levenshtein}
{
	name=distanza di Levenshtein,
	description={misura per la differenza fra due stringhe. In particolare, la distanza di Levenshtein tra due stringhe
	A e B è il numero minimo di modifiche elementari che consentono di trasformare la A nella B. Per modifica elementare si intende la cancellazione di un carattere, la sostituzione di un carattere con un altro o l'inserimento di un carattere.}
}