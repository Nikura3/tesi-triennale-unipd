% !TEX encoding = UTF-8
% !TEX TS-program = pdflatex
% !TEX root = ../tesi.tex

%**************************************************************
\chapter{Descrizione dell'applicativo}
\label{cap:applicazione}
%**************************************************************

In questo capitolo viene descritto l'applicativo le cui funzionalità sono state estese durante lo stage.\\

\section{Architettura serverless}
L'architettura della web app è basata sul \emph{Serverless Framework}, un \gls{framework} che permette di costruire architetture \gls{serverless}. Esso consente inoltre di definire l'architettura \gls{AWS} attraverso un file in formato \gls{YAML}. \\
 Nei paragrafi seguenti vengono mostrati alcuni esempi della sintassi per la definizione delle infrastrutture necessarie ai nuovi servizi implementati ed il procedimento da seguire per effettuare il \gls{deploy} della struttura \gls{serverless}.
	\subsection{Definizione delle resources}
	Attraverso la direttiva \emph{ Resources} è possibile definire le risorse a cui le funzioni \emph{Lambda} possono accedere, ovvero tabelle \emph{DynamoDB} e \emph{bucket S3} .
	
		\subsubsection{Tabelle DynamoDB}
		\begin{figure}[H]
			\centering
			\includegraphics[width=11cm]{immagini/tabellaDB.png} \\
			\caption{\label{fig:tabellaDB} Esempio del codice per la creazione di una tabella DynamoDB}
		\end{figure}
		
		Nel codice sopra riportato:
		\begin{itemize}
			\item \textbf{Type} definisce il tipo di risorsa da creare;
			\item \textbf{TableName} indica il nome della tabella riportato nei servizi \gls{AWS};
			\item \textbf{AttributeDefinitions} descrive gli attributi che compongono la chiave primaria;
			\item \textbf{KeySchema} definisce la struttura della chiave primaria. Nell'esempio la chiave è composta da un solo attributo ma DynamoDB permette di definire anche chiavi più complesse;
			\item \textbf{ProvisionedThroughput} specifica il numero di letture e scritture permesse della risorsa.
		\end{itemize}

		\subsubsection{Bucket S3}
			
		\begin{figure}[H]
			\centering
			\includegraphics[width=11cm]{immagini/bucketS3.png} \\
			\caption{\label{fig:bucketS3} Esempio del codice per la creazione di un bucket S3}
		\end{figure}
	
		Nel codice sopra riportato:
		\begin{itemize}
			\item \textbf{Type} definisce il tipo di risorsa da creare;
			\item \textbf{BucketName} indica il nome del \emph{bucket} riportato nei servizi \gls{AWS};
			\item \textbf{AccessControl} specifica i permessi di accesso al \emph{bucket};
			\item \textbf{LifeCycleConfiguration} permette di definire delle regole per il ciclo di vita degli oggetti all'interno del bucket. Nel caso riportato sono state definite due regole per due diversi prefissi all'interno del \emph{bucket} (\emph{audioExpiration} e \emph{imageExpiration}) entrambe con durata di un giorno;
			\item \textbf{CorsConfiguration} descrive le configurazione per le \gls{CORS} per gli oggetti del \emph{bucket}.
		\end{itemize}
	
	\subsection{Definizione delle funzioni Lambda}
	Per definire le funzioni \emph{Lambda} viene utilizzata la direttiva \emph{functions}. 
	
	\begin{figure}[H]
		\centering
		\includegraphics[width=11cm]{immagini/lambda.png} \\
		\caption{\label{fig:lambda} Esempio del codice per la creazione di una funzione Lambda}
	\end{figure}

	Nel codice sopra riportato:
	\begin{itemize}
		\item \textbf{handler} è il riferimento al file contenente il codice della funzione; 
		\item \textbf{events} indica gli eventi che causano l'esecuzione della funzione \emph{Lambda}. Specificando \item \textbf{Http} permette di definire degli API Gateway HTTP endpoint che quando chiamati provocano l'esecuzione della funzione;
		\item \textbf{path} definisce il path dell'endpoint e identifica la risorsa;
		\item \textbf{method} indica il tipo di accesso HTTP permesso;
		\item \textbf{cors} abilita le \gls{CORS}.
	\end{itemize}
	
	\subsection{Deploy del back-end}

\section{Web-App}
La web app è realizzata utilizzando \emph{Angular 12.x} in collaborazione con \emph{Nebular}, libreria per la realizzazione di interfacce utente. \\
Nei paragrafi successivi vengono descritte le funzionalità già presenti all'interno dell'applicativo prima dell'integrazione e viene mostrato come effettuare il \gls{deploy} delle modifiche effettuate al front-end.
	\subsection{Funzionalità disponibili}
	\subsection{Deploy del front-end}