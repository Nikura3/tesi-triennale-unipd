% !TEX encoding = UTF-8
% !TEX TS-program = pdflatex
% !TEX root = ../tesi.tex

%**************************************************************
\chapter{Descrizione dell'applicativo esistente}
\label{cap:applicazione}
%**************************************************************

In questo capitolo viene descritto l'applicativo preesistente le cui funzionalità sono state estese durante lo stage.\\

\section{Architettura serverless}
L'architettura della web app è basata sul \emph{Serverless Framework}, un \gls{framework} che permette di costruire architetture \gls{serverless}. Esso consente inoltre di definire l'architettura \gls{AWS} attraverso un file in formato \gls{YAML}. \\
 Nei paragrafi seguenti vengono mostrati alcuni esempi della sintassi per la definizione delle infrastrutture necessarie ai nuovi servizi implementati. \\
 Viene inoltre mostrato il procedimento da seguire per effettuare il \gls{deploy} della struttura \gls{serverless}.
	\subsection{Definizione delle resources}
		\subsubsection{Tabelle DynamoDB}
		\subsubsection{Bucket S3}
	\subsection{Definizione delle funzioni Lambda}
	\subsection{Deploy del back-end}

\section{Web-App}
La web app è realizzata utilizzando \emph{Angular 12.x} in collaborazione con \emph{Nebular}, libreria per la realizzazione di interfacce utente. \\
Nei paragrafi successivi vengono descritte le funzionalità già presenti all'interno dell'applicativo prima dell'integrazione e viene mostrato come effettuare il \gls{deploy} delle modifiche effettuate al front-end.
	\subsection{Funzionalità disponibili}
	\subsection{Deploy del front-end}