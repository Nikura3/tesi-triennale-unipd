% !TEX encoding = UTF-8
% !TEX TS-program = pdflatex
% !TEX root = ../tesi.tex

%**************************************************************
\chapter{Conclusioni}
\label{cap:conclusioni}
%**************************************************************

%**************************************************************
\section{Consuntivo finale}

\begin{center}
	
	\renewcommand{\arraystretch}{1.5}
	
	\centering
	\begin{longtable}{| C{2.5cm} | C{2cm} | L{7.2cm} |}
		
		\hline
		
		\rowcolor{lighter-bugBlue}
		\textbf{Durata in ore} & \textbf{Settimana} & \textbf{Descrizione} \\
		
		\hline
		
		40 & 1 &
		\begin{itemize}[leftmargin=*]
			\item Studio delle tecnologie necessarie.
		\end{itemize} \\
		
		\hline
		
		80 & 2, 3 &
		\begin{itemize}[leftmargin=*]
			\item Progettazione e sviluppo di un micro-servizio per attività di \emph{face detection} per 
			la creazione di squadre di gioco;
			\item Integrazione con la piattaforma esistente. 
		\end{itemize}  \\
		
		\hline
		
		40 & 4 &
		\begin{itemize}[leftmargin=*]
			\item Progettazione e sviluppo di una sezione del sito per l'inserimento dei risultati di un nuovo gioco;
			\item Integrazione con la piattaforma esistente. 
		\end{itemize}  \\
		
		\hline
		
		100 & 5, 6, 7 &
		\begin{itemize}[leftmargin=*]
			\item Progettazione e sviluppo di un micro-servizio per il controllo vocale;
			\item Integrazione con la piattaforma esistente. 
		\end{itemize}  \\
		
		\hline 
		
		20 & 7 &
		\begin{itemize}[leftmargin=*]
			\item Progettazione e implementazione di una versione semplificata della pagina per l'inserimento delle 
			partite di Mario Kart da utilizzare in occasione del Summit \gls{AWS} di Milano.
		\end{itemize} \\
		
		\hline
		
		40 & 8 &
		\begin{itemize}[leftmargin=*]
			\item Stesura della documentazione di progetto delle attività di sviluppo condotte nelle settimane precedenti.
		\end{itemize} \\
		
		\hline
		
		\rowcolor{lighter-bugBlue}
		\multicolumn{2}{| c | }{\textbf{Totale ore: }} & 	\multicolumn{1}{  c | }{\textbf{320}}\\
		
		\hline
		
		
		\caption{Consuntivo finale}
	\end{longtable}
	
	
\end{center}

%**************************************************************
\section{Raggiungimento degli obiettivi}
Durante il progetto di stage sono stati soddisfatti tutti i requisiti obbligatori. \\
Il tempo pianificato per la formazione iniziale e la realizzazione del sistema di riconoscimento dei giocatori è stato adeguato. \\
Ha invece richiesto più tempo del previsto l'implementazione del \gls{chatbot} vocale, avendo incontrato alcune difficoltà nell'interpretazione non sempre corretta da parte del bot dei nickname dei giocatori pronunciati dall'utente.
Oltre a quelli classificati obbligatori, sono stati soddisfatti anche tutti i requisiti desiderabili. Sono state quindi aggiunte due pagine al sito: 
\begin{itemize}
	\item Una per l'inserimento dei risultati di gioco a Duck Game;
	\item Una per l'inserimento più rapido delle partite di Mario Kart attraverso riconoscimento facciale. Questa
	funzionalità ha permesso l'utilizzo del prodotto allo stand di \azienda in occasione del Summit \gls{AWS} tenutosi a Milano durante gli ultimi giorni di stage.
\end{itemize}

\noindent Per quanto riguarda invece i requisiti facoltativi, non vi è stato il tempo necessario alla loro implementazione. \\

\noindent Il raggiungimento degli obiettivi, rappresentato dal grado di implementazione dei requisiti, è schematizzato in \autoref{tab:tracciamento} e \autoref{tab:completamento}.

\begin{center}
	
	\renewcommand{\arraystretch}{1.5}
	
	\centering
	\begin{longtable}{| C{2.5cm} | C{4cm} |}
		
		\hline
		
		\rowcolor{lighter-bugBlue}
		\textbf{Requisito} &  \textbf{Stato} \\
		
		\hline
		
		\texttt{RO-1} & Implementato \\
		
		\hline 
		
		\texttt{RO-1.1} & Implementato \\
		
		\hline 
		
		\texttt{RO-1.2} & Implementato \\
		
		\hline 
		
		\texttt{RO-1.3} & Implementato \\
		
		\hline
		
		\texttt{RO-1.4} & Implementato  \\
		
		\hline 
		
		\texttt{RO-1.5} & Implementato \\
		
		\hline 
		
		\texttt{RO-1.6} & Implementato  \\
		
		\hline 
		
		\texttt{RO-2} &Implementato  \\
		
		\hline 
		
		\texttt{RO-2.1} & Implementato  \\
		
		\hline 
		
		\texttt{RO-2.2} & Implementato  \\
		
		\hline
		
		\texttt{RD-1} & Implementato  \\
		
		\hline 
		
		\texttt{RD-1.1} &Implementato  \\
		
		\hline 
		
		\texttt{RD-1.2} & Implementato \\
		
		\hline 
		
		\texttt{RD-1.3} & Implementato  \\	
	
		\hline 
		
		\texttt{RD-2} & Implementato  \\
		
		\hline 
		
		\texttt{RF-1} & Non Implementato \\
		
		\hline 
		
		\caption{Tracciamento dei requisiti}\label{tab:tracciamento}
	\end{longtable}
	
	
\end{center}

\begin{center}
	
	\renewcommand{\arraystretch}{1.5}
	
	\centering
	\begin{longtable}{| C{2.5cm} | C{4cm} |}
		
		\hline
		
		\rowcolor{lighter-bugBlue}
		\textbf{Requisiti} &  \textbf{Completamento} \\
		
		\hline
		
		\textbf{Obbligatori} & 100\% \\
		
		\hline 
		
		\textbf{Desiderabili} & 100\% \\
		
		\hline 
		
		\textbf{Facoltativi} & 0\% \\
		
		\hline 
		
		\caption{Completamento dei requisiti}\label{tab:completamento}
	\end{longtable}
	
	
\end{center}

%**************************************************************
\section{Conoscenze acquisite e valutazione personale}
Durante le otto settimane di stage ho avuto modo di apprendere moltissimo, sia a livello di hard-skill che di soft-skill.

\noindent Ho avuto modo non solo di studiare tecnologie che non avevo mai utilizzato (come ad esempio \gls{AWS} e \emph{Angular}), ma anche di approfondire e affinare le conoscenze che durante questi anni di università avevo acquisito.\\

\noindent Ho imparato inoltre a comunicare in modo più efficace non solo le mie idee ma anche le mie difficoltà, in modo che altri potessero aiutarmi e consigliarmi. Mi ha aiutato anche a comprendere meglio il funzionamento corretto e proficuo di un rapporto collaborativo tra colleghi con lo scopo i raggiungere un obiettivo comune.

\noindent Questo progetto di stage mi ha permesso di sperimentare, mettermi alla prova e utilizzare la mia creatività per 
realizzare un prodotto finale che incontrasse le aspettative dell'azienda. \\

\noindent Valuto dunque molto positivamente l'esperienza di stage svolta presso \azienda poiché mi ha permesso di entrare in contatto con un ambiente lavorativo reale, di mettere a frutto le mie conoscenze pregresse e svilupparne di nuove.
