% !TEX encoding = UTF-8
% !TEX TS-program = pdflatex
% !TEX root = ../tesi.tex

%**************************************************************
\chapter{Descrizione dello stage}
\label{cap:descrizione-stage}
%**************************************************************

%**************************************************************
\section{Introduzione al progetto}
In \azienda è stata creata una piattaforma denominata MariBa con lo scopo di registrare i risultati di gioco del personale a Mario Kart e calcetto balilla. Tale piattaforma è dotata di un sistema di intelligenza artificiale che, in base ai giocatori (o alle coppie nel caso del calcetto), è in grado di predire il risultato del match di gioco. Il limite della
piattaforma attuale è che tutti i dati, dall'inizializzazione di una partita ai risultati finali, devono essere inseriti
manualmente. \\

Al fine di rendere più immediato l'inserimento dei dati si vuole evolvere la piattaforma includendo le seguenti funzionalità:
\begin{itemize}
	\item Sistema di \gls{image recognition} pe riconoscere i giocatori e ruoli durante la fase di inizializzazione della partita
	e formazione delle squadre;
	\item Servizio vocale per l'inserimento dei risultati dei match giocati.
\end{itemize}


%**************************************************************
\section{Obiettivi formativi}
	Gli obiettivi formativi dell'attività di stage sono i seguenti:
	\begin{itemize}
		\item Apprendere come sviluppare un applicativo web con controlli vocali;
		\item Apprendere come svolgere attività di integrazione con servizi di Machine Learning in ambito \gls{image recognition}, \gls{automatic speech recognition} (ASR) e \gls{natural language understanding} (NLU);
	\end{itemize}
%**************************************************************

\section{Requisiti}
Nel primo giorno di stage si è svolto un incontro con il tutor aziendale ed il product owner per definire in modo dettagliato i requisiti. Nel corso dello stage il livello di obbligatorietà di tali requisiti è variato in risposta alle esigenze dell'azienda. \\
 Di seguito viene riportata la versione finale dell'analisi effettuata.

	\subsection{Requisiti obbligatori}
		Di seguito vengono elencati i requisiti obbligatori:
		\begin{itemize}
			\item Sviluppo di un micro-servizio per le attività di \emph{face detection} e \emph{recognition};
			\item Sviluppo di un micro-servizio per le attività di controllo vocale via web
		\end{itemize}
	\subsection{Requisiti desiderabili}
		Di seguito vengono elencati i requisiti desiderabili:
		\begin{itemize}
			\item Integrazione di un nuovo gioco all'interno della piattaforma;
			\item Implementazione di una versione semplificata per l'inserimento dei dati di Mario Kart per l'utilizzo
			della piattaforma in occasione del Summit \gls{AWS} di Milano.
		\end{itemize}
	\subsection{Requisiti facoltativi}
		Di seguito vengono elencati i requisiti facoltativi:
		\begin{itemize}
			\item Sviluppo di una \gls{skill} Alexa con le stesse funzionalità del \gls{chatbot} vocale richiesto come requisito obbligatorio e integrato sulla piattaforma web.
		\end{itemize}

%**************************************************************
\section{Pianificazione}
La durata complessiva dello stage è stata di 8 settimane di lavoro a tempo pieno per un totale di circa 320 ore.
Le attività sono state distribuite nel seguente modo:
\begin{itemize}
	\item una settimana di studio delle tecnologie necessarie;
	\item due settimane di progettazione e sviluppo di un micro servizio per attività di \emph{face detection} per 
	la creazione di squadre di gioco e integrazione con la piattaforma esistente;
	\item una settimana di sviluppo e integrazione nella piattaforma esistente di una sezione del sito per l'inserimento dei risultati di un nuovo gioco;
	\item tre settimane di progettazione e sviluppo di un micro servizio per il controllo vocale e integrazione con la piattaforma esistente;
	\item una settimana di stesura della documentazione di progetto delle attività di sviluppo condotte nelle settimane
	precedenti;
\end{itemize}
Durante tutta la durata dello stage sono stati effettuati stand-up giornalieri con il tutor aziendale in affiancamento per 
monitorare lo stato di avanzamento ed evidenziare eventuali problemi sorti.
