% !TEX encoding = UTF-8
% !TEX TS-program = pdflatex
% !TEX root = ../tesi.tex

%**************************************************************
\chapter{Descrizione dello stage}
\label{cap:descrizione-stage}
%**************************************************************

%**************************************************************
\section{Introduzione al progetto}
In \azienda è stata creata una piattaforma denominata MariBa con lo scopo di registrare i risultati di gioco del personale a Mario Kart e calcetto balilla. Tale piattaforma è dotata di un sistema di intelligenza artificiale che, in base ai giocatori (o alle coppie nel caso del calcetto), è in grado di predire il risultato del match di gioco. Il limite della
piattaforma attuale è che tutti i dati, dall'inizializzazione di una partita ai risultati finali, devono essere inseriti
manualmente. \\

Al fine di rendere più immediato l'inserimento dei dati si vuole evolvere la piattaforma includendo le seguenti funzionalità:
\begin{itemize}
	\item Sistema di \emph{image recognition} per riconoscere i giocatori e definire i loro ruoli durante la fase di inizializzazione della partita
	e formazione delle squadre;
	\item Servizio vocale per l'inserimento dei risultati dei match giocati.
\end{itemize}


%**************************************************************
\section{Obiettivi formativi}
	Gli obiettivi formativi dell'attività di stage sono i seguenti:
	\begin{itemize}
		\item Apprendere come sviluppare un applicativo web con controlli vocali;
		\item Apprendere come svolgere attività di integrazione con servizi di Machine Learning in ambito \emph{image recognition}, \gls{automatic speech recognition} (ASR) e \gls{natural language understanding} (NLU);
	\end{itemize}
%**************************************************************

\section{Requisiti}
Nel primo giorno di stage si è svolto un incontro con il tutor aziendale per definire in modo dettagliato i requisiti funzionali dell'applicativo, elencati in \autoref{tab:requisiti}.\\
Ogni requisito è classificato e identificato univocamente attraverso un codice composto secondo il seguente schema:
\begin{center}
	\texttt{R[Priorità]-[Identificativo]}
\end{center}
Dove:
\begin{itemize}
	\item \textbf{R} indica che si tratta di un requisito;
	\item \textbf{Priorità} può assumere i seguenti valori:
	\begin{itemize}
		\item \textbf{O}: requisito obbligatorio;
		\item \textbf{D}: requisito desiderabile;
		\item \textbf{F}: requisito facoltativo.
	\end{itemize}
	\item \textbf{Identificativo}: numero progressivo che identifica il requisito in forma gerarchica, strutturato
	come segue:
	\begin{center}
		\texttt{[codicePadre].[codiceFiglio]}
	\end{center}
\end{itemize} 

\begin{center}
	
	\renewcommand{\arraystretch}{1.5}
	
	\centering
	\begin{longtable}{| C{3cm} | L{8.7cm} |}
		
		\hline
		
		\rowcolor{lighter-bugBlue}
		\textbf{Codice} &  \textbf{Descrizione} \\
		
		\hline
		
		\texttt{RO-1} & Sviluppo della funzionalità di \emph{face detection} e \emph{recognition} per il riconoscimento automatico dei giocatori. \\
		
		\hline 
		
		\texttt{RO-1.1} & L'utente deve poter utilizzare la webcam per scattare una fotografia ed effettuare il
		riconoscimento dei giocatori in modo automatico. \\
		
		\hline 
		
		\texttt{RO-1.2} & L'utente deve poter associare un volto non riconosciuto ad un nickname esistente. \\
		
		\hline 
		
		\texttt{RO-1.3} & L'utente deve poter creare un nuovo giocatore nel caso non fosse stato riconosciuto dal sistema ed egli non fosse ancora registrato. \\
		
		\hline
		
		\texttt{RO-1.4} & L'utente deve poter selezionare tra i volti individuati nella fotografia quali siano i giocatori che parteciperanno alla partita. \\
		
		\hline 
		
		\texttt{RO-1.5} & L'utente deve poter scegliere la formazione delle squadre per le partite a calcetto balilla. \\
		
		\hline 
		
		\texttt{RO-1.6} & L'utente deve poter utilizzare la funzionalità di predizione del risultato della partita 
		anche in caso di riconoscimento automatico dei giocatori. \\
		
		\hline 
		
		\texttt{RO-2} & Sviluppo di un \gls{chatbot} vocale per permettere l'utilizzo dell'applicativo web attraverso la pronuncia delle attività da svolgere. \\
		
		\hline 
		
		\texttt{RO-2.1} & L'utente deve poter inizializzare una partita utilizzando il \gls{chatbot}. \\
		
		\hline 
		
		\texttt{RO-2.2} & L'utente deve poter salvare i risultati di una partita utilizzando il \gls{chatbot}. \\
		
		\hline
		
		\texttt{RD-1} & Sviluppo di una pagina nel sito esistente per l'inserimento dei risultati per il gioco Duck Game. \\
			
		\hline 
		
		\texttt{RD-1.1} & L'utente può inserire da due a otto giocatori per inizializzare una partita a Duck Game. \\
		
		\hline 
		
		\texttt{RD-1.2} & L'utente può inserire i risultati di una partita a Duck Game e salvarli nel database. \\
		
		\hline 
		
		\texttt{RD-1.3} & L'utente può richiedere la predizione del risultato finale della partita a Duck Game. \\	
			
		\hline 
		
		\texttt{RD-2} & Sviluppo di una versione semplificata della pagina per l'inserimento dei risultati delle 
		partite a Mario Kart. \\
		
		\hline 
		
		\texttt{RF-1} & Sviluppo di una \gls{skill} Alexa avente le stesse funzionalità del \gls{chatbot} vocale 
		richiesto da \texttt{RO-02}. \\
		
		\hline 
		
		\caption{Requisiti}\label{tab:requisiti}
	\end{longtable}
	
	
\end{center}

%**************************************************************
\section{Pianificazione}
La durata complessiva dello stage è stata di 8 settimane di lavoro a tempo pieno per un totale di circa 320 ore. \\

\noindent Secondo il piano di lavoro iniziale definito con l'azienda, le attività sono distribuite come riportato in
\autoref{tab:pianificazione}.

\begin{center}
	
	\renewcommand{\arraystretch}{1.5}
	
		\centering
		\begin{longtable}{| C{2.5cm} | C{2cm} | L{7.2cm} | }
			
			\hline
			
			\rowcolor{lighter-bugBlue}
			\textbf{Durata in ore} & \textbf{Settimana} & \textbf{Descrizione} \\
			
			\hline
			
			40 & 1 &
			\begin{itemize}[leftmargin=*]
				\item Studio delle tecnologie necessarie.
			\end{itemize} \\
			
			\hline
			
			80 & 2, 3 &
			\begin{itemize}[leftmargin=*]
				\item Progettazione e sviluppo di un micro-servizio per attività di \emph{face detection} per 
				la creazione di squadre di gioco;
				\item Integrazione con la piattaforma esistente. 
			\end{itemize}  \\
			
			\hline
		
			
			80 & 4, 5 &
			\begin{itemize}[leftmargin=*]
				\item Progettazione e sviluppo di un micro-servizio per il controllo vocale;
				\item Integrazione con la piattaforma esistente. 
			\end{itemize}  \\
			 
			\hline
			
			80 & 6, 7 &
			\begin{itemize}[leftmargin=*]
				\item Sviluppo della \gls{skill} Alexa per il controllo vocale e l'aggiornamento dei risultati;
				\item Integrazione con la piattaforma esistente. 
			\end{itemize}  \\
			
			\hline
			
			40 & 8 &
			\begin{itemize}[leftmargin=*]
				\item Testing e stesura della documentazione di progetto delle attività di sviluppo condotte nelle settimane precedenti.
			\end{itemize} \\
			
			\hline
			
			\rowcolor{lighter-bugBlue}
			\multicolumn{2}{ | c | }{\textbf{Totale ore: }} & 	\multicolumn{1}{  c | }{\textbf{320}}\\
			
			\hline
		
			
			\caption{Pianificazione delle attività}\label{tab:pianificazione}
		\end{longtable}
		
	
\end{center}
Durante il periodo di stage in azienda il piano di lavoro ha subito modifiche e di conseguenza il consuntivo delle attività svolte riportato nel \autoref{cap:conclusioni} diverge dalla pianificazione qui presentata. Tali modifiche sono state effettuate in risposta a esigenze e richieste dell'azienda. \\
Durante tutta la durata del tirocinio sono stati effettuati stand-up giornalieri con il tutor aziendale per 
monitorare lo stato di avanzamento ed evidenziare eventuali problemi sorti. \\
Al termine dello stage si è svolta una presentazione del prodotto realizzato.
