% !TEX encoding = UTF-8
% !TEX TS-program = pdflatex
% !TEX root = ../tesi.tex

%**************************************************************
\chapter{Descrizione dello stage}
\label{cap:descrizione-stage}
%**************************************************************

%**************************************************************
\section{Introduzione al progetto}
In \azienda è stata creata una piattaforma denominata MariBa con lo scopo di registrare i risultati di gioco del personale a Mario Kart e calcetto balilla. Tale piattaforma è dotata di un sistema di intelligenza artificiale che, in base ai giocatori (o alle coppie nel caso del calcetto), è in grado di predire il risultato del match di gioco. Il limite della
piattaforma attuale è che tutti i dati, dall'inizializzazione di una partita ai risultati finali, devono essere inseriti
manualmente. \\

Al fine di rendere più immediato l'inserimento dei dati si vuole evolvere la piattaforma includendo le seguenti funzionalità:
\begin{itemize}
	\item Sistema di \gls{image recognition} pe riconoscere i giocatori e ruoli durante la fase di inizializzazione della partita
	e formazione delle squadre;
	\item Servizio vocale per l'inserimento dei risultati dei match giocati.
\end{itemize}


%**************************************************************
\section{Obiettivi formativi}
	Gli obiettivi formativi dell'attività di stage sono i seguenti:
	\begin{itemize}
		\item Apprendere come sviluppare un applicativo web con controlli vocali;
		\item Apprendere come svolgere attività di integrazione con servizi di Machine Learning in ambito \gls{image recognition}, \gls{automatic speech recognition} (ASR) e \gls{natural language understanding} (NLU);
	\end{itemize}
%**************************************************************

\section{Requisiti}
Nel primo giorno di stage si è svolto un incontro con il tutor aziendale per definire in modo dettagliato i requisiti. Nel corso dello stage il livello di obbligatorietà di tali requisiti è variato in risposta alle esigenze dell'azienda. \\
 Di seguito viene riportata la versione finale dell'analisi effettuata.

	\subsection{Requisiti obbligatori}
		Di seguito vengono elencati i requisiti obbligatori:
		\begin{itemize}
			\item Sviluppo di un micro-servizio per le attività di \emph{face detection} e \emph{recognition};
			\item Sviluppo di un micro-servizio per le attività di controllo vocale via web
		\end{itemize}
	\subsection{Requisiti desiderabili}
		Di seguito vengono elencati i requisiti desiderabili:
		\begin{itemize}
			\item Integrazione di un nuovo gioco all'interno della piattaforma;
			\item Implementazione di una versione semplificata per l'inserimento dei dati di Mario Kart per l'utilizzo
			della piattaforma in occasione del Summit \gls{AWS} di Milano.
		\end{itemize}
	\subsection{Requisiti facoltativi}
		Di seguito vengono elencati i requisiti facoltativi:
		\begin{itemize}
			\item Sviluppo di una \gls{skill} Alexa con le stesse funzionalità del \gls{chatbot} vocale richiesto come requisito obbligatorio e integrato sulla piattaforma web.
		\end{itemize}

%**************************************************************
\section{Pianificazione}
La durata complessiva dello stage è stata di 8 settimane di lavoro a tempo pieno per un totale di circa 320 ore. \\

\noindent Secondo il piano di lavoro iniziale definito con l'azienda, le attività sono distribuite come segue:

\begin{center}
	
	\renewcommand{\arraystretch}{1.5}
	
		\centering
		\begin{longtable}{| C{2.5cm} | C{2cm} | L{7.2cm} |}
			
			\hline
			
			\rowcolor{lighter-bugBlue}
			\textbf{Durata in ore} & \textbf{Settimana} & \textbf{Descrizione} \\
			
			\hline
			
			40 & 1 &
			\begin{itemize}[leftmargin=*]
				\item Studio delle tecnologie necessarie.
			\end{itemize} \\
			
			\hline
			
			80 & 2, 3 &
			\begin{itemize}[leftmargin=*]
				\item Progettazione e sviluppo di un micro-servizio per attività di \emph{face detection} per 
				la creazione di squadre di gioco;
				\item Integrazione con la piattaforma esistente. 
			\end{itemize}  \\
			
			\hline
		
			
			80 & 4, 5 &
			\begin{itemize}[leftmargin=*]
				\item Progettazione e sviluppo di un micro-servizio per il controllo vocale;
				\item Integrazione con la piattaforma esistente. 
			\end{itemize}  \\
			 
			\hline
			
			80 & 6, 7 &
			\begin{itemize}[leftmargin=*]
				\item Sviluppo della skill Alexa per il controllo vocale e l'aggiornamento dei risultati;
				\item Integrazione con la piattaforma esistente. 
			\end{itemize}  \\
			
			\hline
			
			40 & 8 &
			\begin{itemize}[leftmargin=*]
				\item Testing e stesura della documentazione di progetto delle attività di sviluppo condotte nelle settimane precedenti.
			\end{itemize} \\
			
			\hline
			
			\rowcolor{lighter-bugBlue}
			\multicolumn{2}{ | c | }{\textbf{Totale ore: }} & 	\multicolumn{1}{  c | }{\textbf{320}}\\
			
			\hline
		
			
			\caption{Pianificazione delle attività}
		\end{longtable}
		
	
\end{center}
Durante il periodo di stage in azienda il piano di lavoro ha subito modifiche e di conseguenza la versione finale della pianificazione riportata nel \autoref{cap:conclusioni} diverge da quella qui presentata. Tali modifiche sono state effettuate in risposta alle esigenze e richieste dell'azienda. 
Durante tutta la durata del tirocinio sono stati effettuati stand-up giornalieri con il tutor aziendale in affiancamento per 
monitorare lo stato di avanzamento ed evidenziare eventuali problemi sorti.
