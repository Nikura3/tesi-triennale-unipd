% !TEX encoding = UTF-8
% !TEX TS-program = pdflatex
% !TEX root = ../tesi.tex

%**************************************************************
\chapter{Introduzione}
\label{cap:introduzione}
%**************************************************************

\section{L'azienda}

\azienda è un'azienda informatica nata nel 2012 specializzata nello sviluppo di soluzioni cloud native, sempre in prima linea nel seguire l'evoluzione di questo paradigma tecnologico. \\
L'azienda è partner \gls{AWS} e si occupa di progettazione e sviluppo software Web e Mobile per clienti provenienti da ambiti molto diversificati. \\
L'obiettivo di \azienda è aiutare i propri clienti a definire percorsi di innovazione includendo le tecnologie più avanzate tra cui per esempio
il cloud ed il machine learning per l'analisi di linguaggio naturale, immagini, video e per fare previsioni. 

	\begin{figure}[H]
		\centering
		\includegraphics[width=5cm]{immagini/logo-zero12.png} \\
		\caption{\label{fig:logo_zero12} Logo di Zero12 s.r.l.}
	\end{figure}

%**************************************************************
\section{L'offerta di stage}
	Attualmente in azienda è presente una piattaforma denominata MariBa con lo scopo di registrare i risultati di gioco del personale a Mario Kart e calcetto balilla. L'inserimento di tali dati però è completamente manuale: ogni partita deve essere
	inizializzata con l'inserimento dei nickname di tutti i giocatori e, una volta conclusa, i risultati devono
	essere inseriti manualmente all'interno della piattaforma. L'idea dello stage è di semplificare l'inserimento di questi
	dati attraverso l'utilizzo di tecnologie \gls{AWS} per il riconoscimento automatico dei giocatori e per la registrazione dei risultati comunicandoli vocalmente alla piattaforma. \\
	Il progetto è stato proposto dall'azienda in occasione dell'evento Stage-it 2022 (logo in \autoref{fig:logo_stageit}) finalizzato all'incontro tra aziende e studenti.
	
	\begin{figure}[H]
		\centering
		\includegraphics[width=7cm]{immagini/stageit.png} \\
		\caption{\label{fig:logo_stageit} Logo dell'evento Stage-it 2022}
	\end{figure}

%**************************************************************
\section{Struttura del documento}

	\begin{description}
	    \item[{\hyperref[cap:descrizione-stage]{Il secondo capitolo}}] descrive il progetto di stage e la pianificazione delle attività;
	    
		\item[{\hyperref[cap:tecnologie]{Il terzo capitolo}}] definisce le tecnologie utilizzate durante lo stage;
		
		\item[{\hyperref[cap:applicazione]{Il quarto capitolo}}] descrive il funzionamento generale dell'applicativo integrato;
	    
	    \item[{\hyperref[cap:rekognition]{Il quinto capitolo}}] approfondisce lo sviluppo del sistema di image recognition;
	    
	    \item[{\hyperref[cap:lex]{Il sesto capitolo}}] approfondisce lo sviluppo del sistema di voice service;
	   
	    \item[{\hyperref[cap:conclusioni]{Nel settimo capitolo}}] sono descritte le conclusioni dell'esperienza di stage e gli obiettivi raggiunti.
	\end{description}




